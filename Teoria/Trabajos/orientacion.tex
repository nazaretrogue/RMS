\documentclass[11pt,a4paper]{article}
\usepackage[utf8]{inputenc}
\usepackage[spanish]{babel}	%Idioma
\usepackage{amsmath}
\usepackage{amsfonts}
\usepackage{amssymb}
\usepackage{graphicx} 	%Añadir imágenes
\usepackage{geometry}	%Ajustar márgenes
\usepackage[export]{adjustbox}[2011/08/13]
\usepackage{float}
\restylefloat{table}
\usepackage[hidelinks]{hyperref}
\usepackage{titling}
\graphicspath{{/home/nazaret/Escritorio/LaTEX}}
%\usepackage{minted}
\usepackage{multirow}
\usepackage{caption}
\usepackage{multicol}
\usepackage[shortlabels]{enumitem}
\usepackage{array}
\selectlanguage{spanish}

%Opciones de encabezado y pie de página:
\usepackage{fancyhdr}
\pagestyle{fancy}
\lhead{Nazaret Román Guerrero}
\rhead{Redes Multiservicio}
\lfoot{Grado en Ingeniería Informática}
\cfoot{}
\rfoot{\thepage}
\renewcommand{\headrulewidth}{0.4pt}
\renewcommand{\footrulewidth}{0.4pt}

%Opciones de fuente:
\usepackage[utf8]{inputenc}
\usepackage[default]{sourcesanspro}
\usepackage{sourcecodepro}
\usepackage[T1]{fontenc}

\setlength{\parindent}{15pt}
\setlength{\headheight}{15pt}
\setlength{\voffset}{10mm}

% Custom colors
\usepackage{color}
\definecolor{deepblue}{rgb}{0,0,0.5}
\definecolor{deepred}{rgb}{0.6,0,0}
\definecolor{deepgreen}{rgb}{0,0.5,0}

\usepackage{listings}

\begin{document}
\begin{titlepage}

\begin{minipage}{\textwidth}

\centering
\includegraphics[width=0.5\textwidth]{img/logo.png}\\

\textsc{\Large Redes Multiservicio\\[0.2cm]}
\textsc{GRADO EN INGENIERÍA INFORMÁTICA}\\[1cm]

{\Huge\bfseries Orientación curricular\\}
\noindent\rule[-1ex]{\textwidth}{3pt}\\[3.5ex]
{\large\bfseries Getting hired by a Silicon Valley Company}
\end{minipage}

\vspace{1.5cm}
\begin{minipage}{\textwidth}
\centering

\textbf{Autora}\\ {Nazaret Román Guerrero}\\[2.5ex]
\includegraphics[width=0.3\textwidth]{img/etsiit.jpeg}\\[0.1cm]
\vspace{1cm}
\textsc{Escuela Técnica Superior de Ingenierías Informática y de Telecomunicación}\\
\vspace{1cm}
\textsc{Curso 2018-2019}
\end{minipage}
\end{titlepage}

\pagenumbering{gobble}
\pagenumbering{arabic}

\newpage

\section{Cualidades que debe tener el ingeniero}

Por lo general, a la hora de solicitar un trabajo relacionado con la informática, se dan por supuestos la posesión de

\begin{itemize}
	\item un título de ingeniería informática,
	\item experiencia programando en distintos lenguajes,
	\item capacidad de depuración de código, tráfico de red, sistemas operativos...,
	\item y uso fluido de inglés.
\end{itemize}

\section{¿Qué piden?}

Es importante para la empresa a la que intentas acceder ofrecer una serie de características y valores sobre ti mismo. Entre esos rasgos están:

\begin{itemize}
	\item Actitud del aspirante al trabajo, la curiosidad y las ganas de trabajar que tenga.
	\item El talento y excelencia académicas o en otros trabajos.
	\item Si encaja o no en el equipo de trabajo.
	\item Si el aspirante es capaz o no de comunicarse bien con los demás.
	\item Conocimiento y experiencia en el trabajo al que aspira.
\end{itemize}

Para poder mostrar toda esta información, es necesario hacer uso de un currículum vitae que incluya esta información de distintas formas, de manera que sea fácil de leer, se haga ameno y no se deje información sin redactar. Se pueden utilizar para ello hiperenlaces a una página web personal, utilizar plataformas como GitHub, LinkedIn y stackoverflow y tener cuidado con el uso de las redes sociales y lo que se diga o haga en ellas.\\ 

También es una buena práctica incluir una carta de motivación que exprese tu objetivo al aspirar a ese trabajo, tu interés en él, la ambición, tus metas y resultados...\\

Es importante que el currículum y todos los documentos adjuntos sean objetivos, utilicen un lenguaje formal y educado, muestren la experiencia y proyectos llevados a cabo o en proceso, las habilidades y calificaciones adquiridas y los logros conseguidos a lo largo de la vida laboral y educativa. No se debe incluir información innecesaria ni repetida, ya que no causa una buena impresión.

\end{document}